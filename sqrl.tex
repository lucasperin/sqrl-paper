\documentclass{article}
\usepackage[utf8]{inputenc}
\usepackage[backgroundcolor=yellow]{todonotes}
 
\title{Analysys of SQRL Authentication Protocol using proverif/spass/whatever}
\author{Dettoni, Claudio\\
	\texttt{cd611@cam.ac.uk}
	\and
	Perin, Lucas\\
	\texttt{lucas.perin@posgrad.ufsc.br}
}
\date{December 2016}

\begin{document}

\maketitle

\begin{abstract}
	In this paper we give a simple description of the SQRL authentication
	protocol and an analysys of a known vulnerability. We run an 
	experiment for formal verification of the protocul using ProVerif.
	Furthermore, we analyse a modified version of SQRL proposed to fix
	this vulnerability.
\end{abstract}

\section{Intoduction}
	Authentication protocols play an important role in modern internet
	usability. Most social network providers already have implemented
	some sort of authentication method to make it easy for users to connect
	to different services using the same account and credentials. Facebook \todo{Mention Oauth2a, Single Sign On}
	and google are fine examples of this. However, we know that the good and 
	old password is no longer reliable and has poor usability. \todo{Refs!!}
	Furthermore, most services on the web still use pass phrases for authentication,
	since this has been the main method used in previous decades. Hence,
	a common aproach to provide more security to personal accounts has 
	been the two-step verification, using and One Time Passwords (OTP) 
	provided by a trusted mobile app or token. This	method has gained some 
	space in authentication providers such as the
	ones previously mentioned. Therefore, we now have a mixed authentication process
	where the user can still use his passwords and be prompted for an
	OTP optionally.

	Numerous authentication protocols have been proposed in the past
	decade with the goal of replacing pass phrases with something that
	is actually secure and reliable for authentication providers. One
	good example is the Pico project \todo{ref}, with the main goal of
	completely removing passwords. Other examples of known protocols are ?, ? ?. \todo{enumerate and cite} 
	However, we have yet to see the adoption of one of these methods
	by a popular authentication provider to lead its use across the
	Internet. 

	In this paper, we analyse the SQRL protocol \todo{cite}. This protocol
	has drawed attention since its publication due to its simplicity. However,
	simple does not mean secure. In (?\todo{cite relay attack paper}) the authors
	describe how to use relay attacks to break through the SQRL authentication
	protocol. This attack can also be used with other authentication
	protocols that use images for authentication, such as Pico, ? and ?. \todo{enumerate}.
	Our goal is to use ProVerif, a tool for formal verification of cryptgraphic
	protocols, to demonstrate the SQRL protocol's vulnerability to relay attacks.
	Additionally, we use the same tool to verify a modification to this protocol 
	(proposed by the same author that describe the attack). \todo{conclude something}

\section{Secure Quick Reliable Login Protocol}

	The SQRL protocol, short for Secure Quick Reliable login, is an alternative method 
	for authentication replacing user names and passwords. It uses a QR code
	similar to other protocols, such as the Pico Project and TiQR. To log in
	the user should have a device that is capable of storing offline cryptographic
	keys and compute hash and cryptographic operations. In this work, we will consider that
	this is an application runing in an mobile device, namely the SQRL App. 
	Additionally, denote an existing authentication service provider as ASP. Now
	we can describe the main phases of the protocol as the \emph{Setup, Identity 
	Creation} and \emph{Authentication} phases.

	The \emph{Setup} phase is a standart cryptographic setup procedure where the
	SQRL App generates a random 256 bit master secret (MS). This secret is stored offline and 
	is never to be shared. The following phases will always be derivated of this
	secret, but never give it away. The \emph{Identity Creation} phase and the
	\emph{Authentication} phase are operations that should be syncronized with
	each ASP the user wants to authenticate to. In other words, the user will
	have to create an identity for each ASP and each ASP is responsible 
	for authenticating the user's credential upon log in.

	The \emph{Identity Creation} phase is performed when the user wishes to create an
	identity for a specific ASP. In this step he will create a Domain Secret (DS)
	derivated from the service's domain address and the MS, by using an HMAC function.
	This DS is used only for this specific ASP and no others. Furthermore, the
	MS should remain stored safely in case the user needs to create a new DS. In 
	this same phase, a public key (KU) and a private key (PK) are derived from the
	DS, where the KU is then forwarded to the ASP.

	\todo[inline]{Diagram with Identity creation phase}

	The \emph{Authentication} phase should be the operation that users use the
	most. After the Setup and Identity Creation, the user should have a KU and
	a PK for each of his ASPs. To authenticate the user, the ASP generates a 
	nonce and displays it as a QR code for the user to scan. The user scans the
	nonce using the SQRL App and signs it with the PK. The signature is then
	sent to the ASP from the SQRL APP using a different communication channel,
	that is, a secure connection directly from the SQRL App and the ASP.
	Finally, the ASP verifies the signature received from the SQRL app and 
	approves or rejects the user's login attempt.

	\todo[inline]{Diagram with Authentication phase}


\subsection{Remarks}
	The underlaying cryptographic protocols used to generate random numbers, derivate
	key pairs from master secret and signature schemes are not detailed in this work.
	These details are publicly availabe in (REF) and should not
	affect the attack we analyse here. A more extensive analysis on the security
	of SQRL, by taking these cryptographic algorithms into account, can be found
	in (REF) as well as in discussion groups in the \emph{stack exchange} website.
	\todo[inline]{Update references}

\section{Similar Protocols}
\todo[inline]{OpenID, Pico, TiQR}

\section[Relay attack to visual code authentication schemes]{Relay attack to visual code \\
authentication schemes}
	The relay attack, as describe by the authors in (REF), is not a specific
	attackto the SQRL protocol, but rather to a family of protocol based on
	visual codes. Let us consider then a generic protocol that is compatible
	with the SQRL Authentication phase and other protocols in the same family.

	\todo[inline]{Diagram with the protocol from relay attack paper}

	\todo[inline]{Continue}

	\todo[inline]{Diagram with the proposed fix to the protocols}

\section{Spass/ProVerif implementation}
\todo[inline]{Implementation code and results}

\section{Future Works}
	The experiments results in this work are not original, since the attack
	has been published and the authors are splicitly mentioned in our work.
	However, inspired by this, we can build a tool using spass/proverif to
	help protocol developers to detect the relay attack without too much
	effort.

\section{Final Remarks}
\todo[inline]{Some conclusion}


\end{document}
