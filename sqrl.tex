\documentclass{article}
\usepackage[utf8]{inputenc}
\usepackage[backgroundcolor=yellow]{todonotes}
 
\title{Analysys of SQRL Authentication Protocol using proverif/spass/whatever}
\author{Dettoni, Claudio\\
	\texttt{cd611@cam.ac.uk}
	\and
	Perin, Lucas\\
	\texttt{lucas.perin@posgrad.ufsc.br}
}
\date{December 2016}

\begin{document}

\maketitle

\begin{abstract}
	In this paper we give a full description of the SQRL authentication
	protocol and an analysys of known vulnerabilities. We run an 
	experiment for formal verification of the protocul using ProVerif.
	Furthermore, we analyse a modified version of SQRL proposed to fix
	this vulnerability.
\end{abstract}

\section{Intoduction}
	Authentication protocols play an important role in modern internet
	usability. Most social network providers already have implemented
	some sort of authentication method to make it easy for users to connect
	to different services using the same account and credentials. Facebook
	and google are fine examples of this. However, we know that the good and 
	old ``password" is no longer reliable and has poor usability. \todo{Referencias}
	Most services on the web still use pass phrases for authentication,
	since this has been the main method used in previous decades.

\end{document}
