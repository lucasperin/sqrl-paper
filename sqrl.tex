\documentclass{article}
\usepackage[utf8]{inputenc}
\usepackage[backgroundcolor=yellow]{todonotes}
 
\title{Analysys of SQRL Authentication Protocol using proverif/spass/whatever}
\author{Dettoni, Claudio\\
	\texttt{cd611@cam.ac.uk}
	\and
	Perin, Lucas\\
	\texttt{lucas.perin@posgrad.ufsc.br}
}
\date{December 2016}

\begin{document}

\maketitle

\begin{abstract}
	In this paper we give a full description of the SQRL authentication
	protocol and an analysys of known vulnerabilities. We run an 
	experiment for formal verification of the protocul using ProVerif.
	Furthermore, we analyse a modified version of SQRL proposed to fix
	this vulnerability.
\end{abstract}

\section{Intoduction}
	Authentication protocols play an important role in modern internet
	usability. Most social network providers already have implemented
	some sort of authentication method to make it easy for users to connect
	to different services using the same account and credentials. Facebook \todo{Mention Oauth2, etc?}
	and google are fine examples of this. However, we know that the good and 
	old password is no longer reliable and has poor usability. \todo{Refs!!}
	Furthermore, most services on the web still use pass phrases for authentication,
	since this has been the main method used in previous decades. Hence,
	a common aproach to provide more security to personal accounts has 
	been the two-step verification, using and One Time Passwords (OTP) 
	provided by a trusted mobile app or token. This	method has gained some 
	space in authentication providers such as the
	ones previously mentioned. Therefore, we now have a mixed authentication process
	where the user can still use his passwords and be prompted for an
	OTP optionally.

	Numerous authentication protocols have been proposed in the past
	decade with the goal of replacing pass phrases with something that
	is actually secure and reliable for authentication providers. One
	good example is the Pico project \todo{ref}, with the main goal of
	completely removing passwords. Other examples of known protocols are ?, ? ?. \todo{enumerate and cite} 
	However, we have yet to see the adoption of one of these methods
	by a popular authentication provider to lead on its use across the
	Internet. 

	In this paper, we analyse the SQRL protocol \todo{cite}. This protocol
	has drawed attention since its publication due to its simplicity. However,
	simple does not mean secure. In (?\todo{cite relay attack paper}) the authors
	describe how to use relay attacks to break through the SQRL authentication
	protocol. The attack, however, can be used with other authentication
	protocols that use images for authentication, such as Pico, ? and ?. \todo{enumerate}.
	Our goal is to use ProVerif, a tool for formal verification of cryptgraphic
	protocols, to demonstrate the SQRL protocol's vulnerability to relay attacks.
	Additionally, we use the same tool to verify a modification to this protocol 
	(proposed by the same author that describe the attack). \todo{conclude something}















\end{document}
